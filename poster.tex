\documentclass[final]{beamer}

\mode<presentation> {\usetheme{SSC}}

\usepackage[orientation=portrait,size=a1, margin=0.5cm]{beamerposter}

\usepackage[utf8]{inputenc}
\usepackage[T1]{fontenc}
\usepackage[portuguese]{babel}
\usepackage{natbib}
\usepackage{tikz}
\usepackage{siunitx}
\usepackage{pgfplots}
\usepgfplotslibrary{units}
\pgfplotsset{width=0.3\textwidth}

\title[]{Estruturação do Site e Documentação da Biblioteca de Processamento de Imagens Biomédicas (BIAL)}
\author[Lucas Lellis]{Lucas Santana Lellis}
\institute[ICT - UNIFESP]{Instituto de Ciência e Tecnologia\\Universidade Federal de São Paulo}
\date{Junho, 2015}

\begin{document}
	\begin{frame}[t]{}
		\begin{multicols}{3}
			\section{Overview}
			The goal is to build and validate a spacial drill. It digs by doing a reciprocating
			motion instead of rotation, inspired by the {\em Sir ex Ductile} wasp species.
			\subsection{Why not use standard proven designs?}
			Out of Earth there is no guarantee that gravity will pull down a drill
			strongly enough to make it dig. This design, however, does not need anything pulling it down.
			
			% \begin{figure}[ht]
			%   \begin{center}
			%     \includegraphics[width=0.27\textwidth]{./figures/wasp_drill}
			%     % \includegraphics[width=0.25\textwidth]{./figures/wasp_view}
			%     \caption{Electron microscopy of {\em Sirex Noctilia}'s mandible showing some notable points. \label{fig:sirex:mand}}
			%     %  And a throught view of a female of the species with its sheath visible.}
			%   \end{center}
			% \end{figure}
			
			\subsection{What has been done?}
			Validation of the design's controllability, and implementation of
			a control system. This is what is discussed here.
			
			\section{Implementation}
			The implementation can be logically separated into three parts:
			theoretical calculations, mechanical design and programming.
			
			\subsection{Theoretical calculations}
			The overall rig is composed of the motor plus the test drill rig (See Fig.~\ref{fig:rig}).
			
			\begin{figure}[h]
				\begin{center}
				%	\includegraphics[angle=-90,width=0.25\textwidth]{./figures/rig}
				\end{center}
				\caption{Front view of the test rig with a drill mounted. \label{fig:rig}}
			\end{figure}
			
			The mathematical model of the rig consists of: a standard model of electric motors
			available in the literature, coupled with a specific
			location-based model of the apparatus developed for this analysis.
			
			% Although highly specific to this case. The math "trick" used
			% is valid for almost any setup; hence other rigs can be similarly modelled.
			
			Using this model, the optimum parameters for the system were calculated
			based on the goal that all errors should be minimised.
			The implementation and programming were based in these results.
			
			\vfill
			\columnbreak
			
			\subsection{Mechanical design}
			To assemble the electric motor into the test rig, two mechanical
			parts had to be designed. They can be seen in Figs.~\ref{fig:axis:attach}~and~\ref{fig:ring:attach}.
			
			\begin{figure}[h]
				\begin{center}
				%	\includegraphics[width=0.25\textwidth]{attach_axis_crop}
					\caption{Axis attachment for the test rig. \label{fig:axis:attach}}
				\end{center}
			\end{figure}
			\begin{figure}[h]
				\begin{center}
				%	\includegraphics[width=0.25\textwidth]{attach_ring}
					\caption{Ring attachment for the test rig. \label{fig:ring:attach}}
				\end{center}
			\end{figure}
			
			\subsection{Programming}
			
			With the electro-mechanical apparatus all set, the micro controllers were
			programmed: an Arduino board and an ESCON controller chip. They
			can be seen in Fig.~\ref{fig:control}.
			
			\begin{figure}[h]
				\begin{center}
				%	\includegraphics[angle=-90,width=0.25\textwidth]{./figures/controller}
					\caption{Test rig controllers: Arduino (dark blue) and ESCON (black box) \label{fig:control}}
				\end{center}
			\end{figure}
			
			The ESCON is a high-precision control unit that is not directly
			programmable: it saves and processes all theoretical results for use. The Arduino,
			however, has been fully programmed to interface a computer station with the system
			via software implemented for this purpose (Fig.~\ref{fig:interface}).
			
			\begin{figure}[h]
				\begin{center}
					%\includegraphics[width=0.3\textwidth]{./figures/interface}
					\caption{Drill interface software. \label{fig:interface}}
				\end{center}
			\end{figure}
			
			Also, the Arduino-ESCON programming includes safety procedures
			to shutdown the system whenever the motor is stalling, ensuring safer operating
			conditions.
			
			\section{Results}
			
			The controllability sought was achieved as evidenced in Fig.~\ref{fig:result:1:hertz},
			where the error descends until a near zero value and stabilises about this point.
			
			\begin{figure}[h]
				\begin{center}
				\end{center}
			\end{figure}
			
			The constant rate at which the error decreases guarantees that there will
			be enough time for a failure detection.
			
			\section{Final considerations}
			
			\subsection{Future works}
			\begin{itemize}
				\item Develop a mathematical model for the ground resistance: this can significantly improve control performance.
				\item Further development of the constructed control system and models.
			\end{itemize}
			
			\subsection{Acknowledgements}
			I am very grateful for those who made this opportunity possible for me,
			especially for CNPq and the SWB programme, who enabled this experience, and
			for the STAR Lab, who supplied all the necessary tools for this project.
			
		\end{multicols}
	\end{frame}
\end{document}
